\documentclass[a4paper,12pt]{article}
\usepackage[utf8]{inputenc}
\usepackage[T1]{fontenc}

% for itemize bullet
\usepackage{enumitem}
\setlist[itemize]{label=\textbullet}

% permet calculer textwidth
\usepackage{calc}

% Formats the header
\usepackage{fancyhdr}

% pour plusieurs tailles de caractère
\usepackage[10pt]{moresize}

% permet écrire algorithme
%\usepackage[ruled, vlined, linesnumbered]{algorithm2e}
\usepackage{algorithm}
%\usepackage{algpseudocode}
\usepackage{frpseudocode}


% passe en mode large sur la page A4
\usepackage{a4wide}

% document francisé
\usepackage[french]{babel}

% pour les équations
\usepackage{amsmath}
\usepackage{amssymb}
\usepackage{amsthm}

% pour ajouter les photos
\usepackage{graphicx}
\usepackage{float}
\graphicspath{ {./images/} }

% for create hyperlink step in pdf
\usepackage{hyperref}

\usepackage{varwidth}

% for centering text in table box
\usepackage{array}
\newcolumntype{M}[1]{>{\centering\arraybackslash}m{#1}}

% for quote text
\usepackage{textcomp}
\usepackage{xcolor}
\usepackage{listings}
\usepackage{multirow}
\usepackage[most]{tcolorbox}

\usepackage[backend=biber,style=numeric,sorting=none]{biblatex} %Imports biblatex package
\usepackage{csquotes}
\addbibresource{source.bib} %Import the bibliography file

\setlength{\headheight}{14.49998pt}
\setlength{\topmargin}{-15mm}
\setlength{\textheight}{250mm}
\setlength{\parindent}{0pt} % no paragraph indents
\setlength{\parskip}{15pt}  % distance between 2 paragraph
\setlength{\baselineskip}{5mm}

\newcounter{defcounter}
\newcounter{lemcounter}
\newcounter{comcounter}

\definecolor{lightgray}{gray}{0.85}
\newtcolorbox{definition}[1][]{colback=black!5, colframe=black!5, coltitle=black, fonttitle=\bfseries, title=Définition \arabic{defcounter}, sharp corners, borderline west={4pt}{0pt}{green!80!black}, enhanced, breakable}
\tcbset{colback=lightgray, colframe=lightgray}
\newcommand{\newdefinition}[1]{\addtocounter{defcounter}{1}
\begin{definition}#1\end{definition}}

\newtcolorbox{lemma}[1][]{colback=black!5, colframe=black!5, coltitle=black, fonttitle=\bfseries, title=Lemme \arabic{lemcounter}, sharp corners, borderline west={4pt}{0pt}{red!80!black}, enhanced, breakable}
\tcbset{colback=lightgray, colframe=lightgray}
\newcommand{\newlemma}[1]{\addtocounter{lemcounter}{1}
\begin{lemma}#1\end{lemma}}

\newtcolorbox{commandline}[1][]{colback=black!5, colframe=black!5, coltitle=black, fonttitle=\bfseries, title=Commande \arabic{comcounter}, sharp corners, borderline west={4pt}{0pt}{yellow!95!black}, enhanced, breakable}
\tcbset{colback=lightgray, colframe=lightgray}
\newcommand{\newcommandline}[1]{\addtocounter{comcounter}{1}
\begin{commandline}$\rhd$ #1\end{commandline}}

\newtcolorbox{algorithme}[1][]{colback=black!5, colframe=black!5, sharp corners, top=0mm, toptitle=0mm,
fontupper=\tiny, borderline west={4pt}{0pt}{black!95!black}, enhanced, breakable}
\tcbset{colback=lightgray, colframe=lightgray}
\newcommand{\newalgorithm}[3]{
    \begin{algorithme} \begin{algorithm}[H] \caption{#1} \label{#2}
        \begin{algorithmic} #3 \end{algorithmic} \end{algorithm}
    \end{algorithme}}


\title{\Huge{Projet ``élimination au baseball''}\\
\LARGE{Graphes II et Réseaux}}
\author{\large{Juanfer \textsc{Mercier} -- Adrien \textsc{PICHON}}}
\date{Vendredi 15 Avril 2022}
\makeatletter

\setcounter{tocdepth}{1}

\definecolor{mGreen}{rgb}{0,0.6,0}
\definecolor{mGray}{rgb}{0.5,0.5,0.5}
\definecolor{mPurple}{rgb}{0.58,0,0.82}
\definecolor{backgroundColour}{rgb}{0.95,0.95,0.92}

\lstdefinestyle{JavaStyle}{
    %backgroundcolor=\color{backgroundColour},
    commentstyle=\color{mGreen},
    keywordstyle=\color{magenta},
    numberstyle=\tiny\color{mGray},
    stringstyle=\color{mPurple},
    basicstyle=\footnotesize,
    breakatwhitespace=false,
    breaklines=true,
    captionpos=b,
    keepspaces=true,
    numbers=left,
    numbersep=5pt,
    showspaces=false,
    showstringspaces=false,
    showtabs=false,
    tabsize=2,
    language=Java
}

\begin{document}

\section*{Debrief du vendredi 20/01}

\section{Récapitulatif de la journée}
Nous avons dégagé plusieurs pistes de recherche avec G. Fertin et G. Jean :
\begin{itemize}
    \item[\textbullet] il serait intéressant d'analyser les performances d'une
        première version de la méthode séquentielle décrite dans le debrief du
        vendredi dernier. Cette première version sera naïve et écartera tout
        baitModel trop ambigu. Éventuellement, cette version préliminaire ne
        renverra pas de solution (si tous les baitModels considérés sont
        ambigus alors ils seront tous ignorés). Cette implémentation sera
        à tester sur les 10 000 baits fournis dans le fichier de données.
        Nous cherchons surtout à savoir pour combien de baits nous arrivons
        à obtenir un baitModel ``fusionné''. Lorsqu'un tel baitModel est
        obtenu nous quantifierons sa proximité avec le bait.
    \item un premier état de l'art sur les méthodes d'alignement de séquences
        multiples (ou MSA pour ``Multiple Sequence Alignment'') doit être
        dressé pour discuter de la viabilité de la méthode de fusion de
        baitModels utilisant l'alignement de séquences (voir debrief du
        vendredi 13/01). De plus, des tests utilisant au moins une méthode
        de la littérature sont attendus pour déterminer comment étendre ces
        méthodes à notre problème. \\
\end{itemize}

Dans ce debrief, nous présenterons tout d'abord un état de l'art des méthodes MSA
pour ensuite interpréter les résultats obtenus avec les implémentations de
quelques méthodes de fusion de baitModels.

\section{Alignement de séquences}
Pour déterminer la similarité entre des séquences, les biologistes utilisent
des méthodes d'alignement de séquences. Nous verrons qu'il existe plusieurs
types d'alignements et plusieurs approches pour résoudre le problème
d'alignement. Cependant, ce problème est NP-difficile et la complexité des
algorithmes pour résoudre ce problème à l'optimal est souvent exponentielle.
Dans notre cas, les baitModels sur lesquels nous travaillons sont souvent
courts et peu nombreux. Par conséquent, on peut envisager d'utiliser les
méthodes plus lourdes en termes de calculs mais dont la qualité des solutions
est meilleures.

\subsection{Alignement pair-à-pair}
Dans un premier temps, nous présenterons les méthodes d'alignement de séquences
pair-à-pair. Ces méthodes sont utilisées pour évaluer les similarités entre
deux séquences. Nous serons amenés à fusionner plus de deux baitModels mais les
méthodes pair-à-pair sont utilisées par certaines méthodes d'alignement
multiple et sont, historiquement, les premières méthodes qui ont été étudiées
pour le problème d'alignement.

!! SOURCES !!

-> sous-section sur les types d'alignemnt
-> présentation des algorithmes

\subsection{Alignement multiple}

-> sous-sections sur les différents types d'algos

==> section sur les algos implémentés
==> pq l'alignement local semble être un meilleur choix que l'aligment global

\end{document}
